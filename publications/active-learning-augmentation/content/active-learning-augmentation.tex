\documentclass[parskip=full]{scrartcl}

\pdfoutput=1

\usepackage{breakcites}
\usepackage[square,numbers]{natbib}
\usepackage{float}
\usepackage{graphicx}
\usepackage{geometry}
\geometry{%
	a4paper,
	left=18mm,
	right=18mm,
	top=18mm,
}
\usepackage{amsmath}
\usepackage{enumitem}
\usepackage[ruled,vlined]{algorithm2e}
\usepackage{booktabs}
\usepackage{pgfplotstable}
\pgfplotsset{compat=1.14}
\usepackage{longtable}
\usepackage{tabu}
\usepackage{hyperref}
\date{}

\definecolor{hypecol}{HTML}{0875b7}
\hypersetup{%
    colorlinks,
    linkcolor={hypecol},
    citecolor={hypecol},
    urlcolor={hypecol}
}

\title{%
    Active Learning Augmentation
}

\author{%
	Joao Fonseca\(^{1*}\), Fernando Bacao\(^{1}\)
	\\
	\small{\(^{1}\)NOVA Information Management School, Universidade Nova de Lisboa}
	\\
	\small{*Corresponding Author}
	\\
	\\
	\small{Postal Address: NOVA Information Management School, Campus de
    Campolide, 1070--312 Lisboa, Portugal}
	\\
	\small{Telephone: +351 21 382 8610}
}

\begin{document}

\maketitle

\begin{abstract}
    Abstract goes here.
\end{abstract}

\section{Introduction}

Introduction goes here.

\section{Data Augmentation Methods}

Review on Data Augmentation Methods go here.

\section{Methodology}~\label{sec:methodology}

Methodology goes here.

\subsection{Datasets}~\label{sec:datasets}

Dataset description.

\begin{table}[H]
    \centering
    \addtolength{\leftskip} {-2cm}
    \addtolength{\rightskip}{-2cm}
    \pgfplotstabletypeset[
        col sep=comma,
        string type,
        every head row/.style={%
            before row=\toprule,
            after row=\midrule
        },
        every last row/.style={after row=\bottomrule},
    ]{../analysis/datasets_description.csv}
    \caption{\label{tab:datasets_description}
        Description of the datasets collected from each corresponding scene.
        The sampling strategy is similar to all scenes.
    }
\end{table}

\subsection{Machine Learning Algorithms}~\label{sec:machine_learning_algorithms}

Classifiers and generators used.

\subsection{Evaluation Metrics}~\label{sec:evaluation_metrics}

Performance metrics.

\subsection{Experimental Procedure}~\label{sec:experimental_procedure}

Experimental procedure.

\subsection{Software Implementation}

The experiment was implemented using the Python programming language, along
with the Python libraries
\href{https://scikit-learn.org/stable/}{Scikit-Learn}~\cite{Pedregosa2011},
\href{https://imbalanced-learn.org/en/stable/}{Imbalanced-Learn}~\cite{JMLR:v18:16-365},
\href{https://geometric-smote.readthedocs.io/en/latest/?badge=latest}{Geometric-SMOTE}~\cite{Douzas2019},
\href{https://cluster-over-sampling.readthedocs.io/en/latest/?badge=latest}{Cluster-Over-Sampling}~\cite{Douzas2018},
\href{https://research-learn.readthedocs.io/en/latest/?badge=latest}{Research - Learn}
and
\href{https://mlresearch.readthedocs.io/en/latest/?badge=latest}{ML-Research}
libraries. All functions, algorithms, experiments and results are provided in
the \href{https://github.com/joaopfonseca/ml-research/}{GitHub repository of
the project}.

\section{Results \& Discussion}~\label{sec:results_discussion}

\subsection{Results}~\label{sec:sub_results}


% TODO: Captions need to be rewritten

\begin{table}[H]
    \centering
    \pgfplotstabletypeset[
        col sep=comma,
        string type,
        every head row/.style={%
            before row=\toprule,
            after row=\midrule
        },
        every last row/.style={after row=\bottomrule},
    ]{../analysis/mean_std_aulc_ranks.csv}
    \caption{%
        Mean rankings of the AULC metric over the different datasets (7),
        folds (5) and runs (3) used in the experiment. This means that the use
        of G-SMOTE almost always improves the results of the original
        framework.
    }\label{tab:aulc_ranks}
\end{table}


\begin{table}[htb]
    \centering
    \pgfplotstabletypeset[
        col sep=comma,
        string type,
        every head row/.style={%
            before row=\toprule,
            after row=\midrule
        },
        every last row/.style={after row=\bottomrule},
    ]{../analysis/mean_std_aulc_scores.csv}
    \caption{\label{tab:aulc_scores}
        Average AULC of each AL configuration tested. Each AULC score is
        calculated using the G-mean scores of each iteration in the validation
        set. By the end of the iterative process, each AL configuration used a
        total of 750 instances of the 960 instances that compose the training
        set.
    }
\end{table}


% \captionsetup{justification=centering}
% \pgfplotstabletypeset[
% 	begin table=\begin{longtable},
% 	end table=\end{longtable},
% 	col sep=comma,
% 	header=true,
%     columns={G-mean Score,Classifier,Standard,Proposed}, 
%     string type,
%     every head row/.style={before row=\toprule, after row=\midrule\endhead},
% 	every last row/.style={
%         after row={
%             \bottomrule
%             \caption{
%                 Mean data utilization of AL algorithms, as a percentage of the training set.
%             }\label{tab:optimal_data_utilization}
%         }
%     }
% ]{../analysis/optimal_data_utilization.csv}


\begin{figure}[H]
	\centering
	\includegraphics[width=1\linewidth]{../analysis/data_utilization_rate}
    \caption{%
        Mean data utilization rates. The y-axis shows the percentage of data
        (relative to the baseline AL framework) required to reach the
        different performance thresholds.
    }~\label{fig:dur}
\end{figure}


\begin{table}[H]
    \centering
    \addtolength{\leftskip} {-2cm}
    \addtolength{\rightskip}{-2cm}
    \pgfplotstabletypeset[
        col sep=comma,
        string type,
        every head row/.style={%
            before row=\toprule,
            after row=\midrule
        },
        every last row/.style={after row=\bottomrule},
    ]{../analysis/optimal_mean_std_scores.csv}
    \caption{\label{tab:optimal_mean_std_scores}
        Optimal classification scores. The Maximum Performance (MP)
        classification scores are calculated using classifiers trained using
        the entire training set.
    }
\end{table}

\subsection{Statistical Analysis}~\label{sec:statistical-analysis}


\begin{table}[htb]
	\centering
    \pgfplotstabletypeset[
        col sep=comma,
        string type,
        every head row/.style={%
            before row=\toprule,
            after row=\midrule
        },
        every last row/.style={after row=\bottomrule},
    ]{../analysis/wilcoxon_test.csv}
    \caption{%
    	Adjusted p-values using the Wilcoxon signed-rank method. Bold values
        are statistically significant at a level of $\alpha = 0.05$. The 
        null hypothesis is that the performance of the proposed
        framework is similar to that of the original framework.
    }\label{tab:wilcoxon_test}
\end{table}


\subsection{Discussion}~\label{sec:sub_discussion}

\section{Conclusion}~\label{sec:conclusion}


\bibliography{references}
\bibliographystyle{ieeetr}

\end{document}
