\documentclass[parskip=full]{scrartcl}

\pdfoutput=1

\usepackage{breakcites}
\usepackage[square,numbers]{natbib}
\usepackage{float}
\usepackage{graphicx}
\usepackage{geometry}
\geometry{%
	a4paper,
	left=18mm,
	right=18mm,
	top=18mm,
}
\usepackage{amsmath}
\usepackage{enumitem}
\usepackage[ruled,vlined]{algorithm2e}
\usepackage{booktabs}
\usepackage{pgfplotstable}
\pgfplotsset{compat=1.14}
\usepackage{longtable}
\usepackage{tabu}
\usepackage{hyperref}
\date{}

\definecolor{hypecol}{HTML}{0875b7}
\hypersetup{%
    colorlinks,
    linkcolor={hypecol},
    citecolor={hypecol},
    urlcolor={hypecol}
}

\title{%
    Active Learning Augmentation
}

\author{%
	Joao Fonseca\(^{1*}\), Fernando Bacao\(^{1}\)
	\\
	\small{\(^{1}\)NOVA Information Management School, Universidade Nova de Lisboa}
	\\
	\small{*Corresponding Author}
	\\
	\\
	\small{Postal Address: NOVA Information Management School, Campus de
    Campolide, 1070--312 Lisboa, Portugal}
	\\
	\small{Telephone: +351 21 382 8610}
}

\begin{document}

\maketitle

\begin{abstract}
    Abstract goes here.
\end{abstract}

\section{Introduction}

Introduction goes here.

\section{Data Augmentation Methods}

Review on Data Augmentation Methods go here.

\section{Methodology}~\label{sec:methodology}

Methodology goes here.

\subsection{Datasets}~\label{sec:datasets}

Dataset description.

\subsection{Machine Learning Algorithms}~\label{sec:machine_learning_algorithms}

Classifiers and generators used.

\subsection{Evaluation Metrics}~\label{sec:evaluation_metrics}

Performance metrics.

\subsection{Experimental Procedure}~\label{sec:experimental_procedure}

Experimental procedure.

\subsection{Software Implementation}

The experiment was implemented using the Python programming language, along
with the Python libraries
\href{https://scikit-learn.org/stable/}{Scikit-Learn}~\cite{Pedregosa2011},
\href{https://imbalanced-learn.org/en/stable/}{Imbalanced-Learn}~\cite{JMLR:v18:16-365},
\href{https://geometric-smote.readthedocs.io/en/latest/?badge=latest}{Geometric-SMOTE},
\href{https://cluster-over-sampling.readthedocs.io/en/latest/?badge=latest}{Cluster-Over-Sampling},
\href{https://research-learn.readthedocs.io/en/latest/?badge=latest}{Research-Learn}
and
\href{https://mlresearch.readthedocs.io/en/latest/?badge=latest}{ML-Research}
libraries. All functions, algorithms, experiments and results are provided in
the \href{https://github.com/joaopfonseca/ml-research/}{GitHub repository of
the project}.

\section{Results \& Discussion}~\label{sec:results_discussion}

\subsection{Results}~\label{sec:sub_results}

\subsection{Statistical Analysis}~\label{sec:statistical-analysis}

\subsection{Discussion}~\label{sec:sub_discussion}

\section{Conclusion}~\label{sec:conclusion}


\bibliography{references}
\bibliographystyle{ieeetr}

\end{document}
